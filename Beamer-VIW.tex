\documentclass[10pt]{beamer}

\usetheme[progressbar=frametitle]{metropolis}
\usepackage{appendixnumberbeamer}

\usepackage{booktabs}
\usepackage[scale=2]{ccicons}
\usepackage{tikz}
\usepackage{pgfplots}
\usepgfplotslibrary{dateplot}

\usepackage{xspace}

\definecolor{lightorange}{rgb}{1.,0.7,0.}
\definecolor{lightred}{rgb}{1.,0.,0.7}
\definecolor{thickorange}{rgb}{1.,0.54,0.}
\definecolor{thickgreen}{rgb}{0.6,1.,0.}
\definecolor{thickblue}{rgb}{0.6,0.2,4}
\newcommand{\NodeFill}[3]{\tikz[baseline]{\node[fill=#1!20,anchor=base](#2){#3};}}

\newcommand{\themename}{\textbf{\textsc{metropolis}}\xspace}

\title{2D Viscous internal waves and streaming}
\subtitle{}
% \date{\today}
\date{}
\author{Antoine Renaud}
\institute{Laboratoire de Physique, ENS Lyon}
\titlegraphic{\includegraphics[height=1.5cm]{./graphics/logo_labo-eps-converted-to.pdf}\hfill\includegraphics[height=1.5cm]{./graphics/logo_cnrs_ens-eps-converted-to.pdf}}

\begin{document}

\tikzstyle{every picture}+=[remember picture]
\everymath{\displaystyle}

\maketitle

\section{Introduction}

\begin{frame}[fragile]{Internal waves in labs}

  Viscosity can play an in important role for internal waves generated in labs.The Reynolds number $\mathrm{Re}=\frac{UL}{\nu}$  is several order of magnitude lower  in labs experiments than in the ocean context. 

  Consequently, viscosity associated features might matter :
  \begin{itemize}
    \item Decay of internal wave beams
    \item Boundary layers
    \item Streaming
  \end{itemize}
\end{frame}

\begin{frame}{Table of contents}
  \setbeamertemplate{section in toc}[sections numbered]
  \tableofcontents[hideallsubsections]
\end{frame}

\section{The 2D Boussinesq model}

\begin{frame}[fragile]{2D Boussinesq model : Equations}
  \tikzstyle{na} = [baseline=-.5ex]
  \begin{itemize}
    \item \textbf{Momentum equation:}
      \begin{equation*}
        \partial_{t}\mathbf{u}+\left(\mathbf{u}\cdot\boldsymbol{\nabla}\right)\mathbf{u} = -\boldsymbol{\nabla}P+S\mathbf{e}_{z}+\nu\Delta\mathbf{u}  
      \end{equation*}
    \item \textbf{Buoyancy advection equation}
      \begin{equation*}
        \partial_{t}S+\mathbf{u}\cdot\boldsymbol{\nabla}S =0 
      \end{equation*}
    \item \textbf{Incompressible flow} 
      \begin{equation*}
        \boldsymbol{\nabla}\cdot\mathbf{u} =0
      \end{equation*}
  \end{itemize}
\end{frame}

\begin{frame}[fragile]{2D Boussinesq model : Equations}
  \tikzstyle{na} = [baseline=-.5ex]
  \textbf{Velocity field} : \NodeFill{blue}{}{$\mathbf{u}$}$=\left(u,w\right)$
  \begin{itemize}
    \item \textbf{Momentum equation:}
      \begin{equation*}
        \partial_{t}\NodeFill{blue}{}{$\mathbf{u}$}+\left(\NodeFill{blue}{}{$\mathbf{u}$}\cdot\boldsymbol{\nabla}\right)\NodeFill{blue}{}{$\mathbf{u}$} = -\boldsymbol{\nabla}P+S\mathbf{e}_{z}+\nu\Delta\NodeFill{blue}{}{$\mathbf{u}$}  
      \end{equation*}
    \item \textbf{Buoyancy advection equation}
      \begin{equation*}
        \partial_{t}S+\NodeFill{blue}{}{$\mathbf{u}$}\cdot\boldsymbol{\nabla}S =0 
      \end{equation*}
    \item \textbf{Incompressible flow} 
      \begin{equation*}
        \boldsymbol{\nabla}\cdot\NodeFill{blue}{}{$\mathbf{u}$} =0
      \end{equation*}
  \end{itemize}
\end{frame}

\begin{frame}[fragile]{2D Boussinesq model : Equations}
  \tikzstyle{na} = [baseline=-.5ex]
  \textbf{Nabla operator} \NodeFill{green}{}{$\boldsymbol{\nabla}$}$=\left(\partial_{x},\partial_{z}\right)$ and \textbf{Laplacian operator} \NodeFill{thickgreen}{}{$\Delta$}$=\partial_{x}^{2}+\partial_{z}^{2}$
  \begin{itemize}
    \item \textbf{Momentum equation:}
      \begin{equation*}
        \partial_{t}\mathbf{u}+\left(\mathbf{u}\cdot\NodeFill{green}{}{$\boldsymbol{\nabla}$}\right)\mathbf{u} = -\NodeFill{green}{}{$\boldsymbol{\nabla}$}P+S\mathbf{e}_{z}+\nu\NodeFill{thickgreen}{}{$\Delta$}\mathbf{u}  
      \end{equation*}
    \item \textbf{Buoyancy advection equation}
      \begin{equation*}
        \partial_{t}S+\mathbf{u}\cdot\NodeFill{green}{}{$\boldsymbol{\nabla}$}S =0 
      \end{equation*}
    \item \textbf{Incompressible flow} 
      \begin{equation*}
        \NodeFill{green}{}{$\boldsymbol{\nabla}$}\cdot\mathbf{u} =0
      \end{equation*}
  \end{itemize}
\end{frame}

\begin{frame}[fragile]{2D Boussinesq model : Equations}
  \tikzstyle{na} = [baseline=-.5ex]
  \textbf{Pressure field} \NodeFill{lightorange}{}{$P$} and the \textbf{buoyancy field} \NodeFill{red}{}{$S$}$=-g\frac{\rho-\rho_{0}}{\rho_{0}}$.
  \begin{itemize}
    \item \textbf{Momentum equation:}
      \begin{equation*}
        \partial_{t}\mathbf{u}+\left(\mathbf{u}\cdot\boldsymbol{\nabla}\right)\mathbf{u} = -\boldsymbol{\nabla}\NodeFill{lightorange}{}{$P$}+\NodeFill{red}{}{$S$}\mathbf{e}_{z}+\nu\Delta\mathbf{u}  
      \end{equation*}
    \item \textbf{Buoyancy advection equation}
      \begin{equation*}
        \partial_{t}\NodeFill{red}{}{$S$}+\mathbf{u}\cdot\boldsymbol{\nabla}\NodeFill{red}{}{$S$}=0 
      \end{equation*}
    \item \textbf{Incompressible flow} 
      \begin{equation*}
        \boldsymbol{\nabla}\cdot\mathbf{u} =0
      \end{equation*}
  \end{itemize}
\end{frame}

\begin{frame}{Casimirs, potential energy and background stratification}
	\begin{itemize}
		\item The buoyancy advection equation implies the conservation of Casimirs : 
		\begin{equation*}
			\mathcal{C}_{f}\left[S\right]=\iint S\left(x,z,t\right)\mathrm{d}x\mathrm{d}z
		\end{equation*} where $f$ is any function.
		\item $\implies$ The buyancy field evolves by rearanging itself.
		\item The standard potential energy is defined by : 
		\begin{equation*}
			E_{p}\left[S\right]=-\iint S\left(x,z,t\right)z\;\;\mathrm{d}x\mathrm{d}z
		\end{equation*}
		\item The \NodeFill{yellow}{}{background stratification} is defined by : 
		\begin{equation*}
			E_{p}\left[\NodeFill{yellow}{}{$b_{\mathrm{bg}}$}\right]=\min_{S}\left\{E_p\left[S\right]\left\lvert \mathcal{C}_{f}\left[S\right]=C_{f}\;\;\forall f \right.\right\}
		\end{equation*}
		such that $\NodeFill{yellow}{}{$b_{\mathrm{bg}}\left(x,z,t\right)$} = \intop^{z}\;\NodeFill{red}{}{$N^{2}\left(z'\right)$}\;\mathrm{d}z' $ where \NodeFill{red}{}{$N$} is the local \NodeFill{red}{}{Brunt\--V\"ais\"al\"a frequency}.
	\end{itemize}
\end{frame}

\begin{frame}{Buoyancy and pressure disturbances}
	\begin{itemize}
		\item We define the buyancy disturbance fields by $\NodeFill{yellow}{}{$b$}=S-b_{\mathrm{bg}}$  and the pressure disturbance by $\NodeFill{orange}{}{$p$}=P-\intop^{z}\;b_{\mathrm{bg}}\left(z'\right)\;\mathrm{d}z'$
		\item Injecting this in the equations of motion, one gets :
		\begin{equation*}
			\begin{cases}
				\partial_{t}\mathbf{u}+\left(\mathbf{u}\cdot\boldsymbol{\nabla}\right)\mathbf{u} &= -\boldsymbol{\nabla}\NodeFill{orange}{}{$p$}+\NodeFill{yellow}{}{$b$}\mathbf{e}_{z}+\nu\Delta\mathbf{u}\\
				\partial_{t}\NodeFill{yellow}{}{$b$}+\mathbf{u}\cdot\boldsymbol{\nabla}\NodeFill{yellow}{}{$b$}+\NodeFill{red}{}{$N^{2}\left(z\right)$}w & = 0\\
				\boldsymbol{\nabla}\cdot\mathbf{u}& =0
			\end{cases}
		\end{equation*}
	\end{itemize}
\end{frame}

\begin{frame}{Potential energy of the buyancy disturbance}
	\begin{itemize}
		\item For the standard expression of the potential energy, the fact that $b=0$ is a minimum is not obvious:
		\begin{equation*}
			E_{p}\left[S\right]-E_{p}\left[b_{\mathrm{bg}}\right]=\Delta E_{p}\left[b\right]=-\iint b \;\mathrm{d}x\mathrm{z}
		\end{equation*}
		\item Let us introduce $f=\left(\intop^{z} b_{\mathrm{bg}}\right)^{-1}$, such that :
		\begin{align*}
			\mathcal{C}_{f}\left[b_{\mathrm{bg}}\right]&=\mathcal{C}_{f}\left[b_{\mathrm{bg}}+b\right]\\
			&=\mathcal{C}_{f}\left[b_{\mathrm{bg}} \right]-\Delta E_{p}\left[b\right]+\sum_{k=2}^{\infty}\iint \frac{f^{\left(k\right)}\left(b_{\mathrm{bg}}\right)}{k!}b^{k}\;\;\mathrm{d}x\mathrm{d}z
		\end{align*}
		\item For $\left\lvert b\right\rvert\ll1$ we have 
		\begin{equation*}
			\Delta E_{p}\left[b\right]\approx \iint\frac{b^{2}\left(x,z,t\right)}{2N^{2}\left(z\right)}\mathrm{d}x\mathrm{d}z
		\end{equation*}
		This relation is exact if $N^{2}=\mathrm{Cte}$.
	\end{itemize}
\end{frame}

\begin{frame}[fragile]{2D Boussinesq model : Adimensionalization}
  \tikzstyle{na} = [baseline=-.5ex]
  \begin{itemize}
  	\item We assume $N^{2}=\mathrm{Cte}$ in the following
    \item $\left(\tilde{x},\tilde{z}\right)=K\left(x,z\right)$ where $K$ is a typical wave number (e.g. the wave number of the generator)
    \item $\tilde{t}=\Omega t$ where $\Omega$ is a typical frequency (e.g. the frequency of the generator)
    \item $\tilde{\mathbf{u}}=\frac{K}{\Omega}\mathbf{u}$
    \item $\tilde{b}=\frac{K}{N^{2}}b$
    \item $\tilde{P}=\frac{k^{2}}{\Omega^{2}}P$
  \end{itemize}
\end{frame}

\begin{frame}[fragile]{2D Boussinesq model : Dimensionless parameters \\and adimensionalized equations}
  \tikzstyle{na} = [baseline=-.5ex]
  There are two independant dimensionless parameters :
  \begin{itemize}
    \item The Reynold number : \NodeFill{red}{}{$\mathrm{Re}$}$=\frac{\Omega}{\nu K^{2}}$
    \item The Fround number : \NodeFill{blue}{}{$\mathrm{Fr}$}$=\frac{\Omega}{N}$
  \end{itemize}
  The resulting adimensionalized equations write :
  \begin{equation*}
    \begin{cases}
    \partial_{t}\mathbf{u}+\left(\mathbf{u}\cdot\boldsymbol{\nabla}\right)\mathbf{u} & = -\boldsymbol{\nabla}P+\frac{1}{\NodeFill{blue}{}{$\mathrm{Fr}^{2}$}}b\mathbf{e}_{z}+\frac{1}{\NodeFill{red}{}{$\mathrm{Re}$}}\Delta\mathbf{u} \\ 
    \partial_{t}b+\mathbf{u}\cdot\boldsymbol{\nabla}b+w& =0\\ 
    \boldsymbol{\nabla}\cdot\mathbf{u}& =0
    \end{cases}
  \end{equation*}
\end{frame}

\section{Wave and mean\--flow decomposition}

\begin{frame}[fragile]{Zonally symetric flows}
  We consider here the simplest case of zonally periodic flows :
  \begin{itemize}
    \item The averaging operator is defined by $\overline{u}=\frac{1}{2\pi}\intop_{0}^{2\pi}u\,\mathrm{d}x$.
    \item The wave and mean\--flow decomposition is defined by $\left(u,w,b,P\right)=\left(\overline{u},\overline{w},\overline{b},\overline{P}\right)+\left(u',w',b',P'\right)$
    \item Taking the mean part of the equations of motion leads to the \textbf{mean\--flow equations} :
    \begin{equation*}
    	\begin{cases}
    	\partial_{t}\overline{u}-\frac{1}{\mathrm{Re}}\partial_{zz}\overline{u} &=- \partial_{z}\overline{u'w'}\\
    	\overline{w}&=0\\
    	\overline{P}&=\overline{w'^{2}}\\
    	\partial_{t}\overline{b}&=-\partial_{z}\overline{b'w'}
    	\end{cases}
    \end{equation*}
  \end{itemize}
\end{frame}

\begin{frame}[fragile]{Waves equations}
  \tikzstyle{na} = [baseline=-.5ex]
  \begin{itemize}
  	\item Substracting the mean flow equations to the equations of motions leads to the \textbf{wave equations} :
  	\begin{equation*}
    	\!\!\!\!\!\!\!\!\!\!\!\!\!\!\!\!\!\!\!\!\!\!\!\!\!
    	\begin{cases}
    		\partial_{t}u'+\overline{u}\partial_{x}u'+w'\partial_{z}\overline{u}+\NodeFill{red}{}{$u'\partial_{x}u'+w'\partial_{z}u'-\partial_{z}\overline{u'w'} $} + \partial_{x}P'-\frac{1}{\mathrm{Re}}\Delta u'&=0\\
     		\partial_{t}w'+\overline{u}\partial_{x}w'+\NodeFill{red}{}{$u'\partial_{x}w'+w'\partial_{z}w'-\partial_{z}\overline{w'^{2}}$}+\partial_{z}P'-\frac{1}{\mathrm{Fr}^{2}}b'-\frac{1}{\mathrm{Re}}\Delta w'&=0\\
     		\partial_{t}b'+\overline{u}\partial_{x}b'+\NodeFill{red}{}{$u'\partial_{x}b'+w'\partial_{z}b'-\partial_{z}\overline{b'w'}$}+\underbrace{\left(1+\partial_{z}\overline{b}\right)}_{N^{2}\left(z\right)}w'&=0\\
    		\partial_{x}u'+\partial_{z}w'&=0
   		\end{cases}
  	\end{equation*}
  	\item \NodeFill{red}{}{Non\--linear terms} leading to PSI.
  \end{itemize}
\end{frame}

\section{Inviscid internal waves}

\begin{frame}{Linearized}
	Let us first ignore the non\--linear and viscous terms in the wave equations :
	\begin{equation*}
    	\begin{cases}
    	 	\partial_{t}u'+\overline{u}\partial_{x}u'+w'\partial_{z}\overline{u} + \partial_{x}P'&=0\\
     	 	\partial_{t}w'+\overline{u}\partial_{x}w'+\partial_{z}P'-\frac{1}{\mathrm{Fr}^{2}}b'&=0\\
     		\partial_{t}b'+\overline{u}\partial_{x}b'+N^{2}w'&=0\\
    	 	\partial_{x}u'+\partial_{z}w'&=0
   		\end{cases}
  	\end{equation*}
  	As a direct consequence :
  	\begin{equation*}
  		\begin{cases}
  			\partial_{z}\overline{u'w'}&=\partial_{t}\left(\frac{1}{N^{2}}\overline{b'\left(\partial_{x}w'-\partial_{z}u' \right)}+\frac{1}{N^{4}}\frac{\overline{b'^{2}}}{2}\partial_{zz}\overline{u} \right)\\
  			\overline{b'w'}&=-\partial_{t}\left(\frac{\overline{b'^{2}}}{2N^{2}}\right)
  		\end{cases}
  	\end{equation*}
\end{frame}

\begin{frame}[fragile]{WKB ansatz} 
  \begin{itemize}
  	\item We introduce $a\ll1$ and associated slow variables $\left(Z,T,T_{\mathrm{mf}}\right)=\left(az,at,a^{3}t)\right)$ and assume  $\overline{u}=\overline{u}\left(Z,T_{\mathrm{mf}}\right)=O\left(1\right)$.\\
  	\item WKB ansatz :
  	\begin{equation*}
    	\begin{bmatrix}
    		u\\
    		w\\
    		b\mathrm{Fr}/N\\
    		P
    	\end{bmatrix}
    	=a\sum_{j=0}^{\infty}a^{j}
    	\begin{bmatrix}
    	u_{j}\left(Z,T\right)\\
    	w_{j}\left(Z,T\right)\\
    	b_{j}\left(Z,T\right)\mathrm{Fr}/N\\
    	P_{j}\left(Z,T\right)
    	\end{bmatrix}
    	\exp\left(i\frac{\Phi\left(Z,T\right)}{a}-ikx \right)
  	\end{equation*}
  	\item We define $m=-\partial_{Z}\Phi $, $\omega=\partial_{T}\Phi $ and $\hat{\omega}=\omega-k\overline{u}$
  \end{itemize}
\end{frame}

\begin{frame}[fragile]{WKB ansatz}
	Injecting the WKB ansatz and collecting the first order terms leads to :
  	\begin{equation*}\!\!\!\!\!\!\!\!\!\!
  		\mathbf{M}
  		\begin{bmatrix}
  			u_{0}\\
  			w_{0}\\
  			b_{0}\mathrm{Fr}/N\\
  			P_{0}
  		\end{bmatrix}
  		+a\left(\mathbf{M}
  		\begin{bmatrix}
  			u_{1}\\
  			w_{1}\\
  			b_{1}\mathrm{Fr}/N\\
  			P_{1}
  		\end{bmatrix}+
  		\underbrace{\begin{bmatrix}
  			\partial_{T}u_{0}+w_{0}\partial_{Z}\overline{u} \\
  			\partial_{T}w_{0}+\partial_{Z}P_{0}\\
  			\partial_{T}b_{0}\mathrm{Fr}/N\\
  			\partial_{Z}w_{0}
  		\end{bmatrix}}_{\boldsymbol{T}\left[\overline{u},\phi_{0}\right]}
  	 	\right) =0
  	\end{equation*}
  With : 
  \begin{equation*}
  	\mathbf{M} =
  	\begin{bmatrix}
  		i\hat{\omega} 	&  0  &  0 & -ik  \\
  		0	& i\hat{\omega}  & -\frac{N}{\mathrm{Fr}} & -im\\
  		0   &  \frac{N}{\mathrm{Fr}}  &  i\hat{\omega} & 0\\
  		-ik  &  -im  &  0  &  0
  	\end{bmatrix}
  \end{equation*}
\end{frame}

\begin{frame}[fragile]{Order zero}
  	\begin{equation*}\!\!\!\!\!
  		\mathbf{M}
  		\begin{bmatrix}
  			u_{0}\\
  			w_{0}\\
  			b_{0}\mathrm{Fr}/N\\
  			P_{0}
  		\end{bmatrix}
  		=0 \implies 
  		\begin{cases}
  			\det{\mathbf{M}}&=0\iff \hat{\omega}^{2}=\frac{N^{2}k^{2}}{\mathrm{Fr}^{2}\left(k^{2}+m^{2}\right)}\\
  			\begin{bmatrix}
  				u_{0}\\
  				w_{0}\\
  				b_{0}\mathrm{Fr}/N\\
  				P_{0}
  			\end{bmatrix}&
  			=\phi_{0}\boldsymbol{\mathcal{P}}\\
  			\boldsymbol{\mathcal{P}}&=
  			\begin{bmatrix}
  				\hat{\omega}/k\\
  				-\hat{\omega}/m\\
  				-iN/m\mathrm{Fr}\\
  				\hat{\omega}^{2}/k^{2}
  			\end{bmatrix}
  		\end{cases}
  	\end{equation*}
\end{frame}

\begin{frame}[fragile]{Dispersion relation}
  	\begin{equation*}
  		\hat{\omega}^{2}=\frac{N^{2}k^{2}}{\mathrm{Fr}^{2}\left(k^{2}+m^{2}\right)}
  	\end{equation*}
  	\begin{itemize}
  		\item $-N\leq\mathrm{Fr}\hat{\omega}\leq N$
  		\item Phase and Group velocities : 
  		\begin{equation*}
    		\mathbf{c}_{\varphi}=\frac{\hat{\omega}}{\mathbf{k}^{2}}\mathbf{k}
    		\qquad,\qquad 
    		\mathbf{c}_{g}=-\frac{\hat{\omega}}{\mathbf{k}^{2}}\frac{m}{k}\mathbf{k}^{\bot}
  		\end{equation*}
  		where $\mathbf{k}^{\bot}=\left(-m,k\right)$.
  		\item $\mathbf{c}_{\varphi}\cdot\mathbf{c}_{g}=0$
  	\end{itemize}
\end{frame}

\begin{frame}[fragile]{Order one}
  We take the scalar product of the order one terms with $\phi_{0}^{*}\boldsymbol{\mathcal{P}}^{\dagger}$ : 
  \begin{equation*}
  	\boldsymbol{\mathcal{P}}^{\dagger}\cdot\boldsymbol{\mathcal{T}}\left[\overline{u},\phi_{0} \right]=0
  \end{equation*}
  After some algebra, we get the \textbf{wave\--activity equation} :
  \begin{equation*}
  	\partial_{t}A+\partial_{z}\left(Aw_{g}\right)=0\;\;,\;\;\text{ with }
  	\begin{cases}
  		A&=\frac{E}{\hat{\omega}}\\
  		w_{g}&=-\frac{\mathrm{Fr}^{2}\hat{\omega}^{3} m}{N^{2}k^{2}}
  	\end{cases}
  \end{equation*}
  where 
  \begin{equation*}
  	E=\frac{1}{4}\left(\left\lvert u_{0}\right\rvert^{2}+\left\lvert w_{0}\right\rvert^{2} +\frac{\mathrm{Fr}^{2}}{N^{2}}\left\lvert b_{0}\right\rvert^{2} \right)=\frac{N^{2}\left\lvert \phi_{0}\right\rvert^{2}}{2\mathrm{Fr}^{2}m^{2}}
  \end{equation*}
  We can check that at leading order $\overline{u'w'}=\frac{1}{2}\mathcal{R}\mathrm{e}\left[u_{0}w_{0}^{*} \right]= kAw_{g}$ such that $\partial_{z}\overline{u'w'}\approx-\partial_{t}\left(kA\right)$
\end{frame}

\section{Viscous internal waves}

\begin{frame}{Linearized}
	We put the viscous operator back into the linearized wave euqations :
	\begin{equation*}
    	\begin{cases}
    	 	\partial_{t}u'+\overline{u}\partial_{x}u'+w'\partial_{z}\overline{u} + \partial_{x}P'-\frac{1}{\mathrm{Re}}\Delta u'&=0\\
     	 	\partial_{t}w'+\overline{u}\partial_{x}w'+\partial_{z}P'-\frac{1}{\mathrm{Fr}^{2}}b'-\frac{1}{\mathrm{Re}}\Delta w'&=0\\
     		\partial_{t}b'+\overline{u}\partial_{x}b'+N^{2}w'&=0\\
    	 	\partial_{x}u'+\partial_{z}w'&=0
   		\end{cases}
  	\end{equation*}
  	We still have:
  	\begin{equation*}
  		\overline{b'w'}=-\partial_{t}\left(\frac{\overline{b'^{2}}}{2N^{2}}\right)
  	\end{equation*}\\
   	We use the same assumptions and consider the same WKB ansatz in the following.
\end{frame}

\begin{frame}[fragile]{WKB ansatz}
	Injecting the WKB ansatz and collecting the first order terms leads to :
  	\begin{equation*}\!\!\!\!\!\!\!\!\!\!\!\!\!\!\!\!\!
  		\mathbf{M}
  		\begin{bmatrix}
  			u_{0}\\
  			w_{0}\\
  			b_{0}\mathrm{Fr}/N\\
  			P_{0}
  		\end{bmatrix}
  		+a\left(\mathbf{M}
  		\begin{bmatrix}
  			u_{1}\\
  			w_{1}\\
  			b_{1}\mathrm{Fr}/N\\
  			P_{1}
  		\end{bmatrix}+
  		\underbrace{\begin{bmatrix}
  			\partial_{T}u_{0}+w_{0}\partial_{Z}\overline{u}+\frac{i}{\mathrm{Re}}\left(u_{0}\partial_{Z}m+2m\partial_{Z}u_{0} \right)\\
  			\partial_{T}w_{0}+\partial_{Z}P_{0}+\frac{i}{\mathrm{Re}}\left(w_{0}\partial_{Z}m+2m\partial_{Z}w_{0} \right)\\
  			\partial_{T}b_{0}\mathrm{Fr}/N\\
  			\partial_{Z}w_{0}
  		\end{bmatrix}}_{\boldsymbol{T}\left[\overline{u},\phi_{0}\right]}
  	 	\right) =0
  	\end{equation*}
  With : 
  \begin{equation*}
  	\mathbf{M} =
  	\begin{bmatrix}
  		i\hat{\omega}+\frac{k^{2}+m^{2}}{\mathrm{Re}} 	&  0  &  0 & -ik  \\
  		0	& i\hat{\omega}+\frac{k^{2}+m^{2}}{\mathrm{Re}}  & -\frac{N}{\mathrm{Fr}} & -im\\
  		0   &  \frac{N}{\mathrm{Fr}}  &  i\hat{\omega} & 0\\
  		-ik  &  -im  &  0  &  0
  	\end{bmatrix}
  \end{equation*}
\end{frame}

\begin{frame}[fragile]{Order zero}
  	\begin{equation*}\!\!\!\!\!\!\!\!\!\!\!\!\!
  		\mathbf{M}
  		\begin{bmatrix}
  			u_{0}\\
  			w_{0}\\
  			b_{0}\mathrm{Fr}/N\\
  			P_{0}
  		\end{bmatrix}
  		=0 \implies 
  		\begin{cases}
  			\det{\mathbf{M}}=0&\iff \hat{\omega}\left(\hat{\omega}-i\frac{k^{2}+m^{2}}{\mathrm{Re}}\right)=\frac{N^{2}}{\mathrm{Fr^{2}}}\frac{k^2}{k^2+m^2}\\
  			\begin{bmatrix}
  				u_{0}\\
  				w_{0}\\
  				b_{0}\mathrm{Fr}/N\\
  				P_{0}
  			\end{bmatrix}&
  			=\phi_{0}\boldsymbol{\mathcal{P}}\\
  			\boldsymbol{\mathcal{P}}&=
  			\begin{bmatrix}
  				\hat{\omega}/k\\
  				-\hat{\omega}/m\\
  				-iN/m\mathrm{Fr}\\
  				\frac{\hat{\omega}^{2}}{k^{2}}\left(1-i\frac{\left(k^{2}+m^{2}\right)}{\mathrm{Re}\hat{\omega}} \right)
  			\end{bmatrix}
  		\end{cases}
  	\end{equation*}
\end{frame}

\begin{frame}[fragile]{Dispersion relation}
	$k$ and $\omega$ are fixed by generation process (it has to be zonally symetric for our calculation)
 	\begin{equation*}
    	 \hat{\omega}\left(\hat{\omega}-i\frac{k^{2}+m^{2}}{\mathrm{Re}}\right)=\frac{N^{2}}{\mathrm{Fr^{2}}}\frac{k^2}{k^2+m^2}
  	\end{equation*}
  	We can already remark a few things
  	\begin{itemize}
    	\item $4^{\mathrm{th}}$ order complex polynomial equation for $m$ meaning there are 4 different complex solutions
    	\item The symetry $m\to-m$ indicates two important branches
  	\end{itemize}
  	Two branches :
  	\begin{equation*}
  	  	k^{2}+m^{2}=\frac{\mathrm{Re}\hat{\omega}}{2i}\left(1\pm\sqrt{1-\frac{4ik^{2}N^{2}}{\mathrm{Re}\mathrm{Fr}^{2}\hat{\omega}^{3}}}\right)-1
  	\end{equation*}
\end{frame}

\begin{frame}[fragile]{Large Reynold number limit}
  \tikzstyle{na} = [baseline=-.5ex]
  We now consider large values of the Reynold number (such that $\mathrm{Fr}^2\mathrm{Re}\left\lvert\hat{\omega}\right\rvert^{3}/4k^{2}N^{2}\gg1$). The solution then writes for upwardly propagating waves :
  \begin{align*}
    m_{w}&= -\epsilon m_{0}-\frac{ik^{4}N^{4}}{2\mathrm{Fr^{4}}m_{0}\mathrm{Re}\left\lvert\hat{\omega}\right\rvert^{5}}\\
    m_{bl}&= \left(\epsilon-i\right)\sqrt{\frac{\mathrm{Re}\left\lvert\hat{\omega}\right\rvert}{2}}
  \end{align*}
  where $m_{0}=\sqrt{\frac{N^{2}k^{2}}{\mathrm{Fr}^{2}\hat{\omega}^{2}}-k^{2}}$ is the inviscid value for $m$ and $\epsilon=\mathrm{sign}\left(\hat{\omega}\right)$\\
  Few remarks :
  \begin{itemize}
    \item $m_{w}$ : Propagating branche
    \item $L_{\mathrm{Re}}=2\mathrm{Fr^{4}}\mathrm{Re}m_{0}\left\lvert\hat{\omega}\right\rvert^{5}/k^{4}N^{4}$ : penetration length for the wave beam
    \item $m_{bl}$ : Boundary layer branche
    \item $\delta_{\mathrm{Re}}=\sqrt{2/\mathrm{Re}\left\lvert\hat{\omega}\right\rvert}$ : Boundary layer length
  \end{itemize}
\end{frame}

\section{Boundary conditions}

\begin{frame}{Boundary condition : transverse oscillation}
	Let us consider a horizontally set-up generator. The fluid is viscous wih a \textbf{no\--slip} boundary condition :
	\begin{equation*}
		\mathbf{u}\left(x,z=h_{b}\left(x,t\right),t\right)=\partial_{t}h_{b}\left(x,t\right)\mathbf{e}_{z}
	\end{equation*}
	If we now su hppose that $\left\lvert\left\lvert_{b}\right\rvert\right\rvert\ll1$, we perform the wave and mean\--flow decomposition and linearize this boundary condition to get :
	\begin{equation*}
		\begin{cases}
			\overline{u}\left(z=0,t\right)&=0\\
			u'\left(x,z=0,t\right)&=0\\
			w'\left(x,z=0,t\right)&=\partial_{t}h_{b}\left(x,t\right)
		\end{cases}
	\end{equation*}
	\textbf{Important consequence :}$\intop_{0}^{\infty}\partial_{z}\overline{u'w'}\,\mathrm{d}z=0$.
\end{frame}

\begin{frame}{Boundary condition : Progressive wave}
	Here we consider $h_{b}\left(x,t\right)=a\mathcal{R}\mathrm{e}\left[e^{i\left(t-x\right)}\right]$ corresponding to $\left(\omega,k\right)=\left(1,1\right)$. Considering waves propagating upwardly, using the polarisation :
	\begin{align*}
		&\begin{cases}
			-\left(\frac{\phi_{0,w}\left(0\right)}{m_{w}\left(0\right)}+\frac{\phi_{0,bl}\left(0\right)}{m_{bl}\left(0\right)}\right)&=ia\\
			\phi_{0,w}\left(0\right)+\phi_{0,bl}\left(0\right)&=0\\
		\end{cases}\\
		\implies& 
		\begin{cases}
		\phi_{0,w}\left(0\right)&=-ia\frac{m_{bl}\left(0\right)m_{w}\left(0\right)}{m_{bl}\left(0\right)-m_{w}\left(0\right)}\\
		\phi_{0,bl}\left(0\right)&=ia\frac{m_{bl}\left(0\right)m_{w}\left(0\right)}{m_{bl}\left(0\right)-m_{w}\left(0\right)}
		\end{cases}
	\end{align*}
\end{frame}

\section{Computation of Reynold stress in the large Reynold number limit}

\begin{frame}{General expression}
	\begin{align*}
		\overline{u'w'}=-\frac{\hat{\omega}^{2}}{2}\left(\NodeFill{red}{}{$\left\lvert \phi_{0,w} \right\rvert^{2}\frac{m'_{w}}{\left\lvert m_{w}\right\rvert^{2}}e^{2\int m''_{w}}$}+ \NodeFill{orange}{}{$\left\lvert \phi_{0,bl} \right\rvert^{2}\frac{m'_{bl}}{\left\lvert m_{bl}\right\rvert^{2}}e^{2\int m''_{bl}}$}\right.\;\;\,\\
		+\NodeFill{orange}{}{$\mathcal{R}\mathrm{e}\left[\phi_{0,w}^{*}\phi_{0,bl}\frac{m_{bl}+m_{w}^{*}}{m_{bl}m_{w}}   \right]\mathcal{R}\mathrm{e}\left[e^{i\int\left(m_{w}^{*}-m_{bl} \right)} \right]$}\;\;\,\\
		\left.+\NodeFill{orange}{}{$\mathcal{I}\mathrm{m}\left[\phi_{0,w}^{*}\phi_{0,bl}\frac{m_{bl}+m_{w}^{*}}{m_{bl}m_{w}}   \right]\mathcal{I}\mathrm{m}\left[e^{i\int\left(m_{w}^{*}-m_{bl} \right)} \right]$}  \right)
	\end{align*}
	\begin{itemize}
		\item \NodeFill{red}{}{Bulk streaming}
		\item \NodeFill{orange}{}{Boundary streaming}
	\end{itemize}
\end{frame}

\begin{frame}{Large Reynold limit without mean\--flow}
	\begin{align*}
		\overline{u'w'}=\epsilon\frac{\hat{\omega}^{2}a^{2}}{2}\left\{\frac{1}{m_{0}}\left( \NodeFill{red}{}{$e^{-\frac{z}{2\mathrm{Fr^{4}}m_{0}\mathrm{Re}\left\lvert\hat{\omega}\right\rvert^{5}}}$}-e^{-z\sqrt{\frac{\mathrm{Re}\left\lvert\hat{\omega}\right\rvert}{2}}}\cos{z\sqrt{\frac{\mathrm{Re}\left\lvert\hat{\omega}\right\rvert}{2}}} \right)\right.\\
		-\left.\frac{1}{\sqrt{\mathrm{2Re}\left\lvert\hat{\omega}\right\rvert}} \left(e^{-z\sqrt{2\mathrm{Re}\left\lvert\hat{\omega}\right\rvert}}-e^{-z\sqrt{\frac{\mathrm{Re}\left\lvert\hat{\omega}\right\rvert}{2}}} \left(\cos{z\sqrt{\frac{\mathrm{Re}\left\lvert\hat{\omega}\right\rvert}{2}}}+\sin{z\sqrt{\frac{\mathrm{Re}\left\lvert\hat{\omega}\right\rvert}{2}}} \right) \right)\right\}
	\end{align*}
	\begin{itemize}
		\item \NodeFill{red}{}{Bulk streaming}
		\item We have $\overline{u'w'}\left(z=0\right)=0$
	\end{itemize}
\end{frame}

\begin{frame}{Large Reynold limit \textcolor{orange}{with} mean\--flow for the bulk term}
	\begin{equation*}
		\overline{u'w'}=\epsilon\frac{\hat{\omega}^{2}\left\lvert \phi_{0,w} \right\rvert^{2}}{2m_{0}}e^{-\int \frac{\mathrm{d}z}{2\mathrm{Fr^{4}}m_{0}\mathrm{Re}\left\lvert\hat{\omega}\right\rvert^{5}}}=\epsilon\frac{a^{2}}{2m_{0}\left(0\right)}e^{-\int \frac{\mathrm{d}z}{2\mathrm{Fr^{4}}m_{0}\left(1-\overline{u}\right)\mathrm{Re}\left\lvert1-\overline{u}\right\rvert^{5}}}
	\end{equation*}	\\
	where $\epsilon=\mathrm{sign}\left(1-\overline{u}\right)$
\end{frame}

\begin{frame}{Large Reynold limit \textcolor{orange}{with} mean\--flow for the bulk term : \\ Stading wave forcing}
	We now consider a standing wave forcing corresponding in the superposition of two waves with equal amplitudes and opposite phase speeds :
	\begin{equation*}\!\!\!\!\!\!\!\!\!
		\overline{u'w'}=\frac{a^{2}}{2m_{0}\left(0\right)}\left(\epsilon_{-}e^{-\int \frac{\mathrm{d}z}{2\mathrm{Fr^{4}}m_{0}\left(1-\overline{u}\right)\mathrm{Re}\left\lvert1-\overline{u}\right\rvert^{5}}}-\epsilon_{+}e^{-\int \frac{\mathrm{d}z}{2\mathrm{Fr^{4}}m_{0}\left(1+\overline{u}\right)\mathrm{Re}\left\lvert1+\overline{u}\right\rvert^{5}}} \right)
	\end{equation*}	with $\epsilon_{\pm}=\mathrm{sign}\left(1\pm\overline{u}+\right)$
\end{frame}

\end{document}

