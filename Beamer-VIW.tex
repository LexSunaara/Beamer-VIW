\documentclass[10pt]{beamer}

\usetheme[progressbar=frametitle]{metropolis}
\usepackage{appendixnumberbeamer}

\usepackage{booktabs}
\usepackage[scale=2]{ccicons}
\usepackage{tikz}
\usepackage{pgfplots}
\usepgfplotslibrary{dateplot}

\usepackage{xspace}

\definecolor{lightorange}{rgb}{1.,0.7,0.}
\definecolor{lightred}{rgb}{1.,0.,0.7}
\definecolor{thickorange}{rgb}{1.,0.54,0.}
\definecolor{thickgreen}{rgb}{0.6,1.,0.}
\definecolor{thickblue}{rgb}{0.6,0.2,4}
\newcommand{\NodeFill}[3]{\tikz[baseline]{\node[fill=#1!20,anchor=base](#2){#3};}}

\newcommand{\themename}{\textbf{\textsc{metropolis}}\xspace}

\title{Viscous internal waves and streaming}
\subtitle{}
% \date{\today}
\date{}
\author{Antoine Renaud}
\institute{Laboratoire de Physique, ENS Lyon}
\titlegraphic{\includegraphics[height=1.5cm]{./graphics/logo_labo-eps-converted-to.pdf}\hfill\includegraphics[height=1.5cm]{./graphics/logo_cnrs_ens-eps-converted-to.pdf}}

\begin{document}

\tikzstyle{every picture}+=[remember picture]
\everymath{\displaystyle}

\maketitle

\section{Introduction}

\begin{frame}[fragile]{Internal waves in labs}

  Viscosity can play an in important role for internal waves generated in labs.The Reynolds number $\mathrm{Re}=\frac{UL}{\nu}$  is several order of magnitude lower  in labs experiments than in the ocean context. 

  Consequently, viscosity associated features might matter :
  \begin{itemize}
    \item Decay of internal wave beams
    \item Boundary layers
    \item Streaming
  \end{itemize}
\end{frame}

\begin{frame}{Table of contents}
  \setbeamertemplate{section in toc}[sections numbered]
  \tableofcontents[hideallsubsections]
\end{frame}

\section{The 2D Boussinesq model}

\begin{frame}[fragile]{2D Boussinesq model : Equations}
  \tikzstyle{na} = [baseline=-.5ex]
  \begin{itemize}
    \item \textbf{Momentum equation:}
      \begin{equation*}
        \partial_{t}\mathbf{u}+\left(\mathbf{u}\cdot\boldsymbol{\nabla}\right)\mathbf{u} = -\boldsymbol{\nabla}P+b\mathbf{e}_{z}+\nu\Delta\mathbf{u}  
      \end{equation*}
    \item \textbf{Buoyancy advection equation}
      \begin{equation*}
        \partial_{t}b+\mathbf{u}\cdot\boldsymbol{\nabla}b+N^{2}w =0 
      \end{equation*}
    \item \textbf{Incompressible flow} 
      \begin{equation*}
        \boldsymbol{\nabla}\cdot\mathbf{u} =0
      \end{equation*}
  \end{itemize}
\end{frame}

\begin{frame}[fragile]{2D Boussinesq model : Equations}
  \tikzstyle{na} = [baseline=-.5ex]
  \textbf{Velocity field} : \NodeFill{blue}{}{$\mathbf{u}$}$=\left(u,\NodeFill{thickblue}{}{$w$}\right)$
  \begin{itemize}
    \item \textbf{Momentum equation:}
      \begin{equation*}
        \partial_{t}\NodeFill{blue}{}{$\mathbf{u}$}+\left(\NodeFill{blue}{}{$\mathbf{u}$}\cdot\boldsymbol{\nabla}\right)\NodeFill{blue}{}{$\mathbf{u}$} = -\boldsymbol{\nabla}P+b\mathbf{e}_{z}+\nu\Delta\NodeFill{blue}{}{$\mathbf{u}$}  
      \end{equation*}
    \item \textbf{Buoyancy advection equation}
      \begin{equation*}
        \partial_{t}b+\NodeFill{blue}{}{$\mathbf{u}$}\cdot\boldsymbol{\nabla}b+N^{2}\NodeFill{thickblue}{}{$w$} =0 
      \end{equation*}
    \item \textbf{Incompressible flow} 
      \begin{equation*}
        \boldsymbol{\nabla}\cdot\NodeFill{blue}{}{$\mathbf{u}$} =0
      \end{equation*}
  \end{itemize}
\end{frame}

\begin{frame}[fragile]{2D Boussinesq model : Equations}
  \tikzstyle{na} = [baseline=-.5ex]
  \textbf{Nabla operator} \NodeFill{green}{}{$\boldsymbol{\nabla}$}$=\left(\partial_{x},\partial_{z}\right)$ and \textbf{Laplacian operator} \NodeFill{thickgreen}{}{$\Delta$}$=\partial_{x}^{2}+\partial_{z}^{2}$
  \begin{itemize}
    \item \textbf{Momentum equation:}
      \begin{equation*}
        \partial_{t}\mathbf{u}+\left(\mathbf{u}\cdot\NodeFill{green}{}{$\boldsymbol{\nabla}$}\right)\mathbf{u} = -\NodeFill{green}{}{$\boldsymbol{\nabla}$}P+b\mathbf{e}_{z}+\nu\NodeFill{thickgreen}{}{$\Delta$}\mathbf{u}  
      \end{equation*}
    \item \textbf{Buoyancy advection equation}
      \begin{equation*}
        \partial_{t}b+\mathbf{u}\cdot\NodeFill{green}{}{$\boldsymbol{\nabla}$}b+N^{2}w =0 
      \end{equation*}
    \item \textbf{Incompressible flow} 
      \begin{equation*}
        \NodeFill{green}{}{$\boldsymbol{\nabla}$}\cdot\mathbf{u} =0
      \end{equation*}
  \end{itemize}
\end{frame}

\begin{frame}[fragile]{2D Boussinesq model : Equations}
  \tikzstyle{na} = [baseline=-.5ex]
  \textbf{Pressure field} \NodeFill{lightorange}{}{$P$} and the \textbf{buoyancy field} \NodeFill{red}{}{$b$}$=-g\frac{\rho-\rho_{0}}{\rho_{0}}-$\NodeFill{lightred}{}{$N^{2}$}$z$ where $N$ is the \textbf{Brunt-V\"ais\"al\"a frequency} assumed constant
  \begin{itemize}
    \item \textbf{Momentum equation:}
      \begin{equation*}
        \partial_{t}\mathbf{u}+\left(\mathbf{u}\cdot\boldsymbol{\nabla}\right)\mathbf{u} = -\boldsymbol{\nabla}\NodeFill{lightorange}{}{$P$}+\NodeFill{red}{}{$b$}\mathbf{e}_{z}+\nu\Delta\mathbf{u}  
      \end{equation*}
    \item \textbf{Buoyancy advection equation}
      \begin{equation*}
        \partial_{t}\NodeFill{red}{}{$b$}+\mathbf{u}\cdot\boldsymbol{\nabla}\NodeFill{red}{}{$b$}+\NodeFill{lightred}{}{$N^{2}$}w =0 
      \end{equation*}
    \item \textbf{Incompressible flow} 
      \begin{equation*}
        \boldsymbol{\nabla}\cdot\mathbf{u} =0
      \end{equation*}
  \end{itemize}
\end{frame}

\begin{frame}[fragile]{2D Boussinesq model : Adimensionalization}
  \tikzstyle{na} = [baseline=-.5ex]
  \begin{itemize}
    \item $\left(\tilde{x},\tilde{z}\right)=K\left(x,z\right)$ where $K$ is a typical wave number (e.g. the wave number of the generator)
    \item $\tilde{t}=\Omega t$ where $\Omega$ is a typical frequency (e.g. the frequency of the generator)
    \item $\tilde{\mathbf{u}}=\frac{K}{\Omega}\mathbf{u}$
    \item $\tilde{b}=\frac{K}{N^{2}}b$
    \item $\tilde{P}=\frac{k^{2}}{\Omega^{2}}P$
  \end{itemize}
\end{frame}

\begin{frame}[fragile]{2D Boussinesq model : Dimensionless parameters \\and adimensionalized equations}
  \tikzstyle{na} = [baseline=-.5ex]
  There are two independant dimensionless parameters :
  \begin{itemize}
    \item The Reynold number : \NodeFill{red}{}{$\mathrm{Re}$}$=\frac{\Omega}{\nu K^{2}}$
    \item The Fround number : \NodeFill{blue}{}{$\mathrm{Fr}$}$=\frac{\Omega}{N}$
  \end{itemize}
  The resulting adimensionalized equations write :
  \begin{equation*}
    \begin{cases}
    \partial_{t}\mathbf{u}+\left(\mathbf{u}\cdot\boldsymbol{\nabla}\right)\mathbf{u} & = -\boldsymbol{\nabla}P+\frac{1}{\NodeFill{blue}{}{$\mathrm{Fr}^{2}$}}b\mathbf{e}_{z}+\frac{1}{\NodeFill{red}{}{$\mathrm{Re}$}}\Delta\mathbf{u} \\ 
    \partial_{t}b+\mathbf{u}\cdot\boldsymbol{\nabla}b+w& =0\\ 
    \boldsymbol{\nabla}\cdot\mathbf{u}& =0
    \end{cases}
  \end{equation*}
\end{frame}

\section{Viscous internal waves}

\begin{frame}[fragile]{Linearization}
  \tikzstyle{na} = [baseline=-.5ex]
  Let us linearize the equations of motion about the rest state $\mathrm{\mathbf{u}},b,P=0$ :
  \begin{equation*}
    \begin{cases}
    \partial_{t}u +\partial_{x}P-\frac{1}{\mathrm{Re}}\Delta u& = 0\\ 
    \partial_{t}w +\partial_{z}P-\frac{1}{\mathrm{Fr}^{2}}b-\frac{1}{\mathrm{Re}}\Delta\mathbf{w} & = 0 \\ 
    \partial_{t}b+w& =0\\ 
    \boldsymbol{\nabla}\cdot\mathbf{u}& =0
    \end{cases}
  \end{equation*}
\end{frame}

\begin{frame}[fragile]{Dispersion relation}
  \tikzstyle{na} = [baseline=-.5ex]
  We look for \textbf{non\--vanishing plane waves solutions} $\begin{bmatrix}u\\w\\b\\P\end{bmatrix}=\begin{bmatrix}\tilde{u}\\\tilde{w}\\\tilde{b}\\\tilde{P}\end{bmatrix}e^{\displaystyle i \left(\omega t-k x- m z\right)}$. \\
  This leads to the following \textbf{dispersion relation} :
  \begin{equation*}
    \omega\left(\omega-i\frac{k^{2}+m^{2}}{\mathrm{Re}}\right)=\frac{1}{\mathrm{Fr^{2}}}\frac{k^2}{k^2+m^2}
  \end{equation*}
\end{frame}

\begin{frame}[fragile]{Inviscid limit}
  \tikzstyle{na} = [baseline=-.5ex]
  In the inviscid limit (i.e. $\mathrm{Re}=+\infty$), we recover the well known dispersion relation :
  \begin{equation*}
    \omega^{2}=\frac{1}{\mathrm{Fr^{2}}}\frac{k^2}{k^2+m^2}=\frac{1}{\mathrm{Fr^{2}}}\sin^{2}\theta
  \end{equation*}

  With the phase and group velocities 
  \begin{equation*}
    \mathbf{c}_{\varphi}=\pm \frac{1}{\mathrm{Fr}\left(k^{2}+m^{2}\right)}\begin{bmatrix}k\\m\end{bmatrix} \qquad,\qquad \mathbf{c}_{g}=\pm \frac{k^{2}}{\mathrm{Fr}\sqrt{k^{2}+m{2}}} \begin{bmatrix} m^{2}\\-m k\end{bmatrix} 
  \end{equation*}
  such that $\mathbf{c}_{\varphi}\cdot\mathbf{c}_{g}=0$\\
  We must have $\left\lvert\omega\right\rvert<\frac{1}{\mathrm{Fr}}$ for propagating waves.
\end{frame}

\begin{frame}[fragile]{Back to the viscous case : horizontal generator}
  \tikzstyle{na} = [baseline=-.5ex]
  Let us consider again the viscous case. We consider the case where $\omega=1$ and $k=1$. (generator set\--up horizontally)
  \begin{equation*}
    \mathrm{Fr^{2}}\left(1-i\frac{1+m^{2}}{\mathrm{Re}}\right)\left(1+m^{2}\right)=1
  \end{equation*}
  We can already remark a few things
  \begin{itemize}
    \item $4^{\mathrm{th}}$ order complex polynomial equation for $m$ meaning there are 4 different complex solutions
    \item The symetry $m\to-m$ indicates two important branches
  \end{itemize}
  Two branches :
  \begin{equation*}
    m^{2}=\frac{\mathrm{Re}}{2i}\left(1\pm\sqrt{1-\frac{4i}{\mathrm{Re}\mathrm{Fr}^{2}}}\right)-1
  \end{equation*}
\end{frame}

\begin{frame}[fragile]{Large Reynold number limit}
  \tikzstyle{na} = [baseline=-.5ex]
  We now consider large values of the Reynold number (such that $\mathrm{Fr}^2\mathrm{Re}\gg1$). The solution then writes :
  \begin{align*}
    m_{w}&=\pm\left(m_{0}+\frac{i}{2\mathrm{Fr^{4}}m_{0}\mathrm{Re}}\right)\\
    m_{bl}&=\pm \left(1-i\right)\sqrt{\frac{\mathrm{Re}}{2}}
  \end{align*}
  where $m_{0}=\sqrt{\frac{1}{\mathrm{Fr}^{2}}-1}$ is the inviscid value for $m$.\\
  Few remarks :
  \begin{itemize}
    \item $m_{w}$ : Propagating branche
    \item $L_{\mathrm{Re}}=2\mathrm{Fr^{4}}\mathrm{Re}m_{0}$ : penetration length for the wave beam
    \item $m_{bl}$ : Boundary layer branche
    \item $\delta_{\mathrm{Re}}=\sqrt{2/\mathrm{Re}}$ : Boundary layer length
  \end{itemize}
\end{frame}

\begin{frame}[fragile]{Back to the viscous case : vertical generator}
  \tikzstyle{na} = [baseline=-.5ex]
  We consider here the case where $\omega=1$ and $m=1$. (generator set\--up vertically)
  \begin{equation*}
    \mathrm{Fr^{2}}\left(1+k^{2}\right)\left(1-i\frac{1+k^{2}}{\mathrm{Re}}\right)-k^{2}=0
  \end{equation*}
  Two branches :
  \begin{equation*}
    k^{2}=\frac{i\mathrm{Re}}{2k_{0}^{2}}\left(1+\frac{2ik_{0}^{2}}{\mathrm{Re}}-\sqrt{1+4i\frac{\left(1+k_{0}^{2}\right)k_{0}^{2}}{\mathrm{Re}}} \right)
  \end{equation*}
  where $k_{0}=\frac{\mathrm{Fr}}{\sqrt{1-\mathrm{Fr}^{2}}}$ is the inviscid value for $k$.
\end{frame}

\begin{frame}[fragile]{Large Reynold number limit}
  \tikzstyle{na} = [baseline=-.5ex]
  We now consider large values of the Reynold number. The solution then writes :
  \begin{align*}
    k_{w}&=\pm\left(k_{0}-\frac{ik_{0}\left(1+k_{0}^{2}\right)^{2}}{2\mathrm{Re}}\right)\\
    m_{bl}&=\pm \left(1+i\right)\sqrt{\frac{\mathrm{Re}}{2k_{0}^{2}}}
  \end{align*}
  where $k_{0}=\frac{\mathrm{Fr}}{\sqrt{1-\mathrm{Fr}^{2}}}$ is the inviscid value for $k$.\\
  Few remarks :
  \begin{itemize}
    \item $k_{w}$ : Propagating branche
    \item $L_{\mathrm{Re}}=\frac{2\mathrm{Re}}{k_{0}\left( 1+k_{0}^{2}\right)^{2}}$ : Penetration length for the wave beam
    \item $k_{bl}$ : Boundary layer branche
    \item $\delta_{\mathrm{Re}}=\sqrt{2k^{2}_{0}/\mathrm{Re}}$ : Boundary layer length
  \end{itemize}
\end{frame}

\section{Streaming}

\begin{frame}[fragile]{Wave\--mean flow decomposition}
  \tikzstyle{na} = [baseline=-.5ex]
  \begin{itemize}
    \item The averaging operator is defined by $\overline{u}=\frac{1}{\left(2\pi\right)^{2}}\intop_{0}^{2\pi}\intop_{0}^{2\pi}u\,\mathrm{d}x\mathrm{d}t$.
    \item The wave\--mean decomposition is defined by $\left(u,w,b,P\right)=\left(\overline{u},\overline{w},\overline{b},\overline{P}\right)+\left(u',w',b',P'\right)$
    \item Taking the mean part of the equations of motion leads to :\begin{equation*}\partial_{t}\overline{u}-\frac{1}{\mathrm{Re}}\Delta\overline{u} =- \partial_{z}\overline{u'w'}\end{equation*} and $\overline{w},\overline{b}=0$.
    \item Streaming is induced by the waves from the Reynold stress $\partial_{z}\overline{u'w'}$
  \end{itemize}
\end{frame}

\begin{frame}[fragile]{Waves equations}
  \tikzstyle{na} = [baseline=-.5ex]
  \begin{equation*}
    \!\!\!\!\!\!\!\!\!\!\!\!\!\!
    \begin{cases}
      \partial_{t}u'+\overline{u}\partial{x}u'+w'\partial_{z}\overline{u}+\NodeFill{red}{}{$u'\partial_{x}u'+w'\partial_{z}u'-\partial_{z}\overline{u'w'} $}& = -\partial_{x}P'+\frac{1}{\mathrm{Re}}\Delta u'\\
      \partial_{t}w'+\overline{u}\partial{x}w'+\NodeFill{red}{}{$u'\partial_{x}w'+w'\partial_{z}w'-\partial_{z}\overline{w'^{2}}$}&=-\partial_{z}P'+\frac{1}{\mathrm{Fr}^{2}}b'+\frac{1}{\mathrm{Re}}\Delta w'\\
      \partial_{t}b'+\overline{u}\partial_{x}b'+\NodeFill{red}{}{$u'\partial_{x}b'+w'\partial_{z}b'$}+w'&=0\\
      \partial_{x}u'+\partial_{z}w'&=0
    \end{cases}
  \end{equation*}\\
  \NodeFill{red}{}{Non\--linear terms} responsible of the PSI.
\end{frame}

\section{Waves in shear\--flows : WKB solutions}

\begin{frame}[fragile]{WKB ansatz and linearization}
  \tikzstyle{na} = [baseline=-.5ex]
  We introduce a small dimensionles parameter $a\ll1$ and assume that the mean\--flow writes $U=U\left(Z,T\right)$ where $\left(Z,T\right)=a\left(z,t\right)$.\\
  WKB ansatz :
  \begin{equation*}
    \begin{bmatrix}u\\w\\b\\P\end{bmatrix}=\sum_{j=0}^{\infty}a^{j+1}\begin{bmatrix}u_{j}\left(Z,T\right)\\w_{j}\left(Z,T\right)\\b_{j}\left(Z,T\right)\\P_{j}\left(Z,T\right)\end{bmatrix}\exp\left(i\frac{\Phi\left(Z,T\right)}{a}-ix \right)
  \end{equation*}
  Injecting this ansatz into the wave equation and collecting the leading order terms in $a$ leads to :
  \begin{equation*}\!\!\!\!\!
  	\mathbf{M}\begin{bmatrix}u_{0}\\w_{0}\\b_{0}\\P_{0}\end{bmatrix}+a\left(M\begin{bmatrix}u_{1}\\w_{1}\\b_{1}\\P_{1}\end{bmatrix}+
  	\begin{bmatrix}
  	\partial_{T}u_{0}+w_{0}\partial_{Z}U+\frac{i}{\mathrm{Re}}\left(u_{0}\partial_{Z}m+2m\partial_{Z}u_{0} \right)\\
  	\partial_{T}w_{0}+\partial_{Z}P_{0}+\frac{i}{\mathrm{Re}}\left(w_{0}\partial_{Z}m+2m\partial_{Z}w_{0} \right)\\
  	\partial_{T}b_{0}\\
  	\partial_{Z}w_{0}\end{bmatrix} \right) =0
  \end{equation*}
\end{frame}

\begin{frame}[fragile]{WKB ansatz and linearization}
  \tikzstyle{na} = [baseline=-.5ex]
  \begin{equation*}\!\!\!\!\!\!\!\!\!\!
  	\mathbf{M}
  	\begin{bmatrix}
  		u_{0}\\
  		w_{0}\\
  		b_{0}\\
  		P_{0}
  	\end{bmatrix}+a\left(M
  	\begin{bmatrix}
  		u_{1}\\
  		w_{1}\\
  		b_{1}\\
  		P_{1}
  	\end{bmatrix}+
  	\begin{bmatrix}
  		\partial_{T}u_{0}+w_{0}\partial_{Z}U+\frac{i}{\mathrm{Re}}\left(u_{0}\partial_{Z}m+2m\partial_{Z}u_{0} \right)\\
  		\partial_{T}w_{0}+\partial_{Z}P_{0}+\frac{i}{\mathrm{Re}}\left(w_{0}\partial_{Z}m+2m\partial_{Z}w_{0} \right)\\
  		\partial_{T}b_{0}\\
  		\partial_{Z}w_{0}
  	\end{bmatrix}
  	 \right) =0
  \end{equation*}
  With : 
  \begin{align*}
  	\mathbf{M} &=
  	\begin{bmatrix}
  		i\left(\omega-U\right)+\frac{1+m^{2}}{\mathrm{Re}} 	&  0  &  0 & -i  \\
  		0	& i\left(\omega-U\right)+\frac{1+m^{2}}{\mathrm{Re}}  & -\frac{1}{\mathrm{Fr}^{2}} & -im\\
  		0   &  1  &  i\left(\omega-U\right) & 0\\
  		-i  &  -im  &  0  &  0
  	\end{bmatrix}\\
  	\omega & = \partial_{T}\Phi \\
  	m &= -\partial_{Z}\Phi
  \end{align*}
\end{frame}

\begin{frame}[fragile]{Order zero}
  \tikzstyle{na} = [baseline=-.5ex]
  \begin{equation*}\!\!\!\!\!
  	\mathbf{M}
  	\begin{bmatrix}
  		u_{0}\\
  		w_{0}\\
  		b_{0}\\
  		P_{0}
  	\end{bmatrix}
  	=0 \implies 
  	\begin{cases}
  		\det{\mathbf{M}}&=0\\
  		\begin{bmatrix}
  			u_{0}\\
  			w_{0}\\
  			b_{0}\\
  			P_{0}
  		\end{bmatrix}&
  		=\phi_{0}\boldsymbol{\mathcal{P}}\\
  		\boldsymbol{\mathcal{P}}&=
  		\begin{bmatrix}
  			U-\omega\\
  			\frac{\omega-U}{m}\\
  			\frac{i}{m}\\
  			-\left(\omega-U\right)^{2}\left(1-i\frac{1+m^{2}}{\mathrm{Re}\left(\omega-U\right)}\right)
  		\end{bmatrix}
  	\end{cases}
  \end{equation*}
\end{frame}

\begin{frame}[fragile]{Order one}
  \tikzstyle{na} = [baseline=-.5ex]
  \begin{align*}\!\!\!\!\!\!\!\!\!\!
  	\begin{bmatrix}
  		U-\omega\\
  		\frac{\omega-U}{m}\\
  		-\frac{i}{m\mathrm{Fr}^{2}}\\
  		-\left(\omega-U\right)^{2}\left(1-i\frac{1+m^{2}}{\mathrm{Re}\left(\omega-U\right)}\right)
  	\end{bmatrix}
  		\cdot
  	\begin{bmatrix}
  		\partial_{T}u_{0}+w_{0}\partial_{Z}U+\frac{i}{\mathrm{Re}}\left(u_{0}\partial_{Z}m+2m\partial_{Z}u_{0} \right)\\
  		\partial_{T}w_{0}+\partial_{Z}P_{0}+\frac{i}{\mathrm{Re}}\left(w_{0}\partial_{Z}m+2m\partial_{Z}w_{0} \right)\\
  		\partial_{T}b_{0}\\
  		\partial_{Z}w_{0}
  	\end{bmatrix}\\
  	=0\\
  	\implies \mathcal{F}\left[U,\phi_{0} \right]=0\;\;\;\;\;\;\;\;\;\;\;\;\;\;\;\;\;\;\;\;\;\;\;\;\;\;\;\;\;\;\;\;\;\;\;\;\;\;\;\;\;\;\;\;\;\;\;\;\;\;\;
  \end{align*}
  where $\mathcal{F}$ is differential operator (linear in $\phi_{0}$).
\end{frame}

\begin{frame}[fragile]{Inviscid limit}
  \tikzstyle{na} = [baseline=-.5ex]
  For $\mathrm{Re}=\infty$, the last equation can be simplified into the \textbf{wave activity equation} :
  \begin{equation*}
  	\partial_{T}A+\partial_{Z}\left(A w_{g}\right)=0
  \end{equation*}
  with $A=E/\left(\omega-U\right)$ and $E=\frac{1}{4}\left(\left\lvert u_{0}\right\rvert^{2}+\left\lvert w_{0}\right\rvert^{2}+\mathrm{Fr}^{2}\left\lvert b_{0}\right\rvert^{2}\right)$. \\
  Also $\overline{u_{0}w_{0}}=Akw_{g}$ such that at leading order :
  \begin{equation*}
  	\partial_{z}\overline{u_{0}w_{0}}=-\partial_{t}\left(kA\right)
  \end{equation*}
  Injecting this result into the mean\--flow evolution equation leads to 
  \begin{equation*}
  	\partial_{t}\left(U-kA\right)=0
  \end{equation*}
  This result is known as the \textbf{non-acceleration theorem}.
\end{frame}

\section{Boundary conditions: computation of the full wave field}

\begin{frame}{Boundary condition : transverse oscillation}
	Let us consider a horizontally set-up generator. The fluid is viscous wih a \textbf{no\--slip} boundary condition :
	\begin{equation*}
		\mathbf{u}\left(x,z=h_{b}\left(x,t\right),t\right)=\partial_{t}h_{b}\left(x,t\right)\mathbf{e}_{z}
	\end{equation*}
	If we now suppose that $\left\lvert\left\lvert h_{b}\right\rvert\right\rvert\ll1$, we perform the wave\--decomposition and linearize this boundary condition to get :
	\begin{equation*}
		\begin{cases}
			\overline{u}\left(z=0,t\right)&=0\\
			u'\left(x,z=0,t\right)&=0\\
			w'\left(x,z=0,t\right)&=\partial_{t}h_{b}\left(x,t\right)
		\end{cases}
	\end{equation*}
	\textbf{Important consequence :}$\intop_{0}^{\infty}\partial_{z}\overline{u'w'}\,\mathrm{d}z=0$.
\end{frame}

\begin{frame}{Boundary condition : Progressive wave}
	Here we consider $h_{b}\left(x,t\right)=\epsilon\mathcal{R}\mathrm{e}\left[e^{i\left(t-x\right)}\right]$ corresponding to $\left(\omega,k\right)=\left(1,1\right)$. Considering waves propagating upwardly, the we retain the solution for $m$ with a negative imaginary part only. We first ignore the mean\--flow. 
	\begin{align*}
		&\begin{cases}
			\tilde{w}'\left(z\right)&=a_{w}\left(Z\right)e^{-i\intop_{0}^{z}m_{w}\,\mathrm{d}z}+a_{bl}\left(Z\right)e^{-i\intop_{0}^{z}m_{bl},\mathrm{d}z}\\
			\partial_{z}\tilde{w}'\left(z=0\right)&=0\\
			\tilde{w}'\left(z=0\right)&=i\epsilon
		\end{cases}\\
		\implies& 
		\begin{cases}
		a_{w}\left(0\right)&=i\epsilon\frac{m_{bl}\left(0\right)}{m_{bl}\left(0\right)-m_{w}\left(0\right)}\\
		a_{bl}\left(0\right)&=i\epsilon\frac{m_{w}\left(0\right)}{m_{w}\left(0\right)-m_{bl}\left(0\right)}\\
		\end{cases}
	\end{align*}
\end{frame}

\section{Computation of Reynold stress in large Reynold number limit}

\begin{frame}{General expression}
	For a wave field of the form $\tilde{w}\left(z\right)=a_{w}e^{-i\int m_{w}}+a_{bl}e^{-i\int m_{bl}}$, we have :
	\begin{align*}
		\overline{u'w'}=-\frac{1}{2}\left(\NodeFill{red}{}{$\left\lvert a_{w} \right\rvert^{2}m'_{w}e^{2\int m''_{w}}$}+ \NodeFill{orange}{}{$\left\lvert a_{bl} \right\rvert^{2}m'_{bl}e^{2\int m''_{bl}}$}\right.\;\;\,\\
		+\NodeFill{orange}{}{$\mathcal{R}\mathrm{e}\left[a_{w}^{*}a_{bl}\left(m_{bl}+m_{w}^{*} \right)  \right]\mathcal{R}\mathrm{e}\left[e^{i\int\left(m_{w}^{*}-m_{bl} \right)} \right]$}\;\;\,\\
		\left.+\NodeFill{orange}{}{$\mathcal{I}\mathrm{m}\left[ a_{w}^{*}a_{bl}\left(m_{bl}+m_{w}^{*}\right) \right]\mathcal{I}\mathrm{m}\left[e^{i\int\left(m_{w}^{*}-m_{bl} \right)} \right]$}  \right)
	\end{align*}
	\begin{itemize}
		\item \NodeFill{red}{}{Bulk streaming}
		\item \NodeFill{orange}{}{Boundary streaming}
	\end{itemize}
\end{frame}

\begin{frame}{Large Reynold limit without mean\--flow}
	\begin{align*}
		\overline{u'w'}=-\frac{\epsilon^{2}}{2}\left\{m_{0}\left( \NodeFill{red}{}{$e^{-\frac{z}{\mathrm{Fr}^{4}m_{0}\mathrm{Re}}}$}-e^{-z\sqrt{\frac{\mathrm{Re}}{2}}}\cos{z\sqrt{\frac{\mathrm{Re}}{2}}} \right)\right.\\
		-\left.\frac{m_{0}^{2}}{\sqrt{\mathrm{2Re}}} \left(e^{-z\sqrt{2\mathrm{Re}}}-e^{-z\sqrt{\frac{\mathrm{Re}}{2}}} \left(\cos{z\sqrt{\frac{\mathrm{Re}}{2}}}+\sin{z\sqrt{\frac{\mathrm{Re}}{2}}} \right) \right)\right\}
	\end{align*}
	\begin{itemize}
		\item \NodeFill{red}{}{Bulk streaming}
		\item We have $\overline{u'w'}\left(z=0\right)=0$
	\end{itemize}
\end{frame}

\begin{frame}{Large Reynold limit \textcolor{orange}{with} mean\--flow for the bulk term}
	\begin{equation*}
		\overline{u'w'}=-\frac{\epsilon^{2} m_{0} \left(\overline{u}\right)}{2\left(1-\overline{u}\right)^{2} } e^{-\int\frac{\mathrm{d}z}{\mathrm{Fr}^{4}\left(1-\overline{u}\right)^{5} m_{0}\left(\overline{u}\right)\mathrm{Re}}}
	\end{equation*}
	\begin{equation*}
		\partial_{z}\overline{u'w'}=\frac{\epsilon^{2}}{2\left(1-\overline{u}\right)^{7}\mathrm{Fr}^{4}\mathrm{Re} } e^{-\int\frac{\mathrm{d}z}{\mathrm{Fr}^{4}\left(1-\overline{u}\right)^{5} m_{0}\left(\overline{u}\right)\mathrm{Re}}}
	\end{equation*}
\end{frame}

\end{document}

