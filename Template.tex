\documentclass[10pt]{beamer}

\usetheme[progressbar=frametitle]{metropolis}
\usepackage{appendixnumberbeamer}

\usepackage{booktabs}
\usepackage[scale=2]{ccicons}
\usepackage{tikz}
\usepackage{pgfplots}
\usepgfplotslibrary{dateplot}

\usepackage{xspace}

\definecolor{lightorange}{rgb}{1.,0.7,0.}
\definecolor{lightred}{rgb}{1.,0.,0.7}
\definecolor{thickorange}{rgb}{1.,0.54,0.}
\definecolor{thickgreen}{rgb}{0.6,1.,0.}
\definecolor{thickblue}{rgb}{0.6,0.2,4}
\newcommand{\hllight}[1]{\sethlcolor{lightorange}\hl{#1}}
\newcommand{\hlthick}[1]{\sethlcolor{thickorange}\hl{#1}}
\newcommand{\NodeFill}[3]{\tikz[baseline]{\node[fill=#1!20,anchor=base](#2){#3};}}

\newcommand{\themename}{\textbf{\textsc{metropolis}}\xspace}

\title{Viscous internal waves and streaming}
\subtitle{}
% \date{\today}
\date{}
\author{Antoine Renaud}
\institute{Laboratoire de Physique, ENS Lyon}
\titlegraphic{\includegraphics[height=1.5cm]{./graphics/logo_labo-eps-converted-to.pdf}\hfill\includegraphics[height=1.5cm]{./graphics/logo_cnrs_ens-eps-converted-to.pdf}}

\begin{document}

\tikzstyle{every picture}+=[remember picture]
\everymath{\displaystyle}

\maketitle

\section{Introduction}

\begin{frame}[fragile]{Internal waves in labs}

  Viscosity can play an in important role for internal waves generated in labs.The Reynolds number $\mathrm{Re}=\frac{UL}{\nu}$  is several order of magnitude lower  in labs experiments than in the ocean context. 

  Consequently, viscosity associated features might matter :
  \begin{itemize}
    \item Decay of internal wave beams
    \item Boundary layers
    \item Streaming
  \end{itemize}
\end{frame}

\begin{frame}{Table of contents}
  \setbeamertemplate{section in toc}[sections numbered]
  \tableofcontents[hideallsubsections]
\end{frame}
\section{The 2D Boussinesq model}

\begin{frame}[fragile]{2D Boussinesq model : Equations}
  \tikzstyle{na} = [baseline=-.5ex]
  \begin{itemize}
    \item \textbf{Momentum equation:}
      \begin{equation*}
        \partial_{t}\mathbf{u}+\left(\mathbf{u}\cdot\boldsymbol{\nabla}\right)\mathbf{u} = -\boldsymbol{\nabla}P+b\mathbf{e}_{z}+\nu\Delta\mathbf{u}  
      \end{equation*}
    \item \textbf{Buoyancy advection equation}
      \begin{equation*}
        \partial_{t}b+\mathbf{u}\cdot\boldsymbol{\nabla}b+N^{2}w =0 
      \end{equation*}
    \item \textbf{Incompressible flow} 
      \begin{equation*}
        \boldsymbol{\nabla}\cdot\mathbf{u} =0
      \end{equation*}
  \end{itemize}
\end{frame}

\begin{frame}[fragile]{2D Boussinesq model : Equations}
  \tikzstyle{na} = [baseline=-.5ex]
  \textbf{Velocity field} : \NodeFill{blue}{}{$\mathbf{u}$}$=\left(u,\NodeFill{thickblue}{}{$w$}\right)$
  \begin{itemize}
    \item \textbf{Momentum equation:}
      \begin{equation*}
        \partial_{t}\NodeFill{blue}{}{$\mathbf{u}$}+\left(\NodeFill{blue}{}{$\mathbf{u}$}\cdot\boldsymbol{\nabla}\right)\NodeFill{blue}{}{$\mathbf{u}$} = -\boldsymbol{\nabla}P+b\mathbf{e}_{z}+\nu\Delta\NodeFill{blue}{}{$\mathbf{u}$}  
      \end{equation*}
    \item \textbf{Buoyancy advection equation}
      \begin{equation*}
        \partial_{t}b+\NodeFill{blue}{}{$\mathbf{u}$}\cdot\boldsymbol{\nabla}b+N^{2}\NodeFill{thickblue}{}{$w$} =0 
      \end{equation*}
    \item \textbf{Incompressible flow} 
      \begin{equation*}
        \boldsymbol{\nabla}\cdot\NodeFill{blue}{}{$\mathbf{u}$} =0
      \end{equation*}
  \end{itemize}
\end{frame}

\begin{frame}[fragile]{2D Boussinesq model : Equations}
  \tikzstyle{na} = [baseline=-.5ex]
  \textbf{Nabla operator} \NodeFill{green}{}{$\boldsymbol{\nabla}$}$=\left(\partial_{x},\partial_{z}\right)$ and \textbf{Laplacian operator} \NodeFill{thickgreen}{}{$\Delta$}$=\partial_{x}^{2}+\partial_{z}^{2}$
  \begin{itemize}
    \item \textbf{Momentum equation:}
      \begin{equation*}
        \partial_{t}\mathbf{u}+\left(\mathbf{u}\cdot\NodeFill{green}{}{$\boldsymbol{\nabla}$}\right)\mathbf{u} = -\NodeFill{green}{}{$\boldsymbol{\nabla}$}P+b\mathbf{e}_{z}+\nu\NodeFill{thickgreen}{}{$\Delta$}\mathbf{u}  
      \end{equation*}
    \item \textbf{Buoyancy advection equation}
      \begin{equation*}
        \partial_{t}b+\mathbf{u}\cdot\NodeFill{green}{}{$\boldsymbol{\nabla}$}b+N^{2}w =0 
      \end{equation*}
    \item \textbf{Incompressible flow} 
      \begin{equation*}
        \NodeFill{green}{}{$\boldsymbol{\nabla}$}\cdot\mathbf{u} =0
      \end{equation*}
  \end{itemize}
\end{frame}

\begin{frame}[fragile]{2D Boussinesq model : Equations}
  \tikzstyle{na} = [baseline=-.5ex]
  \textbf{Pressure field} \NodeFill{lightorange}{}{$P$} and the \textbf{buoyancy field} \NodeFill{red}{}{$b$}$=-g\frac{\rho-\rho_{0}}{\rho_{0}}-$\NodeFill{lightred}{}{$N^{2}$}$z$ where $N$ is the \textbf{Brunt-V\"ais\"al\"a frequency} assumed constant
  \begin{itemize}
    \item \textbf{Momentum equation:}
      \begin{equation*}
        \partial_{t}\mathbf{u}+\left(\mathbf{u}\cdot\boldsymbol{\nabla}\right)\mathbf{u} = -\boldsymbol{\nabla}\NodeFill{lightorange}{}{$P$}+\NodeFill{red}{}{$b$}\mathbf{e}_{z}+\nu\Delta\mathbf{u}  
      \end{equation*}
    \item \textbf{Buoyancy advection equation}
      \begin{equation*}
        \partial_{t}\NodeFill{red}{}{$b$}+\mathbf{u}\cdot\boldsymbol{\nabla}\NodeFill{red}{}{$b$}+\NodeFill{lightred}{}{$N^{2}$}w =0 
      \end{equation*}
    \item \textbf{Incompressible flow} 
      \begin{equation*}
        \boldsymbol{\nabla}\cdot\mathbf{u} =0
      \end{equation*}
  \end{itemize}
\end{frame}

\begin{frame}[fragile]{2D Boussinesq model : Adimensionalization}
  \tikzstyle{na} = [baseline=-.5ex]
  \begin{itemize}
    \item $\left(\tilde{x},\tilde{z}\right)=K\left(x,z\right)$ where $K$ is a typical horizontal wave number (e.g. the wave number of the generator)
    \item $\tilde{t}=\Omega t$ where $\Omega$ is a typical frequency (e.g. the frequency of the generator)
    \item $\tilde{\mathbf{u}}=\frac{K}{\Omega}\mathbf{u}$
    \item $\tilde{b}=\frac{K}{N^{2}}b$
    \item $\tilde{P}=\frac{k^{2}}{\Omega^{2}}P$
  \end{itemize}
\end{frame}

\begin{frame}[fragile]{2D Boussinesq model : Dimensionless parameters \\and adimensionalized equations}
  \tikzstyle{na} = [baseline=-.5ex]
  There are two independant dimensionless parameters :
  \begin{itemize}
    \item The Reynold number : \NodeFill{red}{}{$\mathrm{Re}$}$=\frac{\Omega}{\nu K^{2}}$
    \item The Fround number : \NodeFill{blue}{}{$\mathrm{Fr}$}$=\frac{\Omega}{N}$
  \end{itemize}
  The resulting adimensionalized equations write :
  \begin{equation*}
    \begin{cases}
    \partial_{t}\mathbf{u}+\left(\mathbf{u}\cdot\boldsymbol{\nabla}\right)\mathbf{u} & = -\boldsymbol{\nabla}P+\frac{1}{\NodeFill{blue}{}{$\mathrm{Fr}^{2}$}}b\mathbf{e}_{z}+\frac{1}{\NodeFill{red}{}{$\mathrm{Re}$}}\Delta\mathbf{u} \\ 
    \partial_{t}b+\mathbf{u}\cdot\boldsymbol{\nabla}b+w& =0\\ 
    \boldsymbol{\nabla}\cdot\mathbf{u}& =0
    \end{cases}
  \end{equation*}
\end{frame}

\section{Viscous internal waves}

\begin{frame}[fragile]{Linearization}
  \tikzstyle{na} = [baseline=-.5ex]
  Let us linearize the equations of motion about the rest state $\mathrm{\mathbf{u}},b,P=0$ :
  \begin{equation*}
    \begin{cases}
    \partial_{t}u +\partial_{x}P-\frac{1}{\mathrm{Re}}\Delta u& = 0\\ 
    \partial_{t}w +\partial_{z}P-\frac{1}{\mathrm{Fr}^{2}}b-\frac{1}{\mathrm{Re}}\Delta\mathbf{w} & = 0 \\ 
    \partial_{t}b+w& =0\\ 
    \boldsymbol{\nabla}\cdot\mathbf{u}& =0
    \end{cases}
  \end{equation*}
\end{frame}

\begin{frame}[fragile]{Dispersion relation}
  \tikzstyle{na} = [baseline=-.5ex]
  We look for \textbf{non\--vanishing plane waves solutions} $\begin{bmatrix}u\\w\\b\\P\end{bmatrix}=\begin{bmatrix}\tilde{u}\\\tilde{w}\\\tilde{b}\\\tilde{P}\end{bmatrix}e^{\displaystyle i \left(\omega t-k x- m z\right)}$. \\
  This leads to the following \textbf{dispersion relation} :
  \begin{equation*}
    \omega\left(\omega-i\frac{k^{2}+m^{2}}{\mathrm{Re}}\right)=\frac{1}{\mathrm{Fr^{2}}}\frac{k^2}{k^2+m^2}
  \end{equation*}
\end{frame}

\begin{frame}[fragile]{Inviscid limit}
  \tikzstyle{na} = [baseline=-.5ex]
  In the inviscid limit (i.e. $\mathrm{Re}=+\infty$), we recover the well known dispersion relation :
  \begin{equation*}
    \omega^{2}=\frac{1}{\mathrm{Fr^{2}}}\frac{k^2}{k^2+m^2}=\frac{1}{\mathrm{Fr^{2}}}\sin^{2}\theta
  \end{equation*}

  With the phase and group velocities 
  \begin{equation*}
    \mathbf{c}_{\varphi}=\pm \frac{1}{\mathrm{Fr}\left(k^{2}+m^{2}\right)}\begin{bmatrix}k\\m\end{bmatrix} \qquad,\qquad \mathbf{c}_{g}=\pm \frac{k^{2}}{\mathrm{Fr}\sqrt{k^{2}+m{2}}} \begin{bmatrix} m^{2}\\-m k\end{bmatrix} 
  \end{equation*}
  such that $\mathbf{c}_{\varphi}\cdot\mathbf{c}_{g}=0$\\
  We must have $\left\lvert\omega\right\rvert<\frac{1}{\mathrm{Fr}}$ for propagating waves.
\end{frame}

\begin{frame}[fragile]{Back to the viscous case}
  \tikzstyle{na} = [baseline=-.5ex]
  Let us consider again the viscous case. We consider the case where $\omega=1$ and $k=1$. (generator set\--up horizontally)
  \begin{equation*}
    \mathrm{Fr^{2}}\left(1-i\frac{1+m^{2}}{\mathrm{Re}}\right)\left(1+m^{2}\right)=1
  \end{equation*}
  We can already remark a few things
  \begin{itemize}
    \item $4^{\mathrm{th}}$ order complex polynomial equation for $m$ meaning there are 4 different complex solutions
    \item The symetry $m\to-m$ indicates two important branches
  \end{itemize}
  Two branches :
  \begin{equation*}
    m^{2}=\frac{\mathrm{Re}}{2i}\left(1\pm\sqrt{1-\frac{4i}{\mathrm{Re}\mathrm{Fr}^{2}}}\right)-1
  \end{equation*}
\end{frame}

\begin{frame}[fragile]{Large Reynold number limit}
  \tikzstyle{na} = [baseline=-.5ex]
  We now consider large values of the Reynold number (such that $\mathrm{Fr}^2\mathrm{Re}\gg1$). The solution then writes :
  \begin{align*}
    m_{w}&=\pm\left(m_{0}+\frac{i}{2\mathrm{Fr^{4}}m_{0}\mathrm{Re}}\right)\\
    m_{bl}&=\pm \left(1-i\right)\sqrt{\frac{\mathrm{Re}}{2}}
  \end{align*}
  where $m_{0}=\sqrt{\frac{1}{\mathrm{Fr}^{2}}-1}$ is the inviscid value for $m$.\\
  Few remarks :
  \begin{itemize}
    \item $m_{w}$ : Propagating branche
    \item $L_{\mathrm{Re}}=2\mathrm{Fr^{4}}\mathrm{Re}m_{0}$ : Dumping scale for the wave beam
    \item $m_{bl}$ : Boundary layer branche
    \item $\delta_{\mathrm{Re}}=\sqrt{2/\mathrm{Re}}$ : Boundary layer length
  \end{itemize}
\end{frame}


\begin{frame}{Metropolis titleformats}
	\themename supports 4 different titleformats:
	\begin{itemize}
		\item Regular
		\item \textsc{Smallcaps}
		\item \textsc{allsmallcaps}
		\item ALLCAPS
	\end{itemize}
	They can either be set at once for every title type or individually.
\end{frame}

{
    \metroset{titleformat frame=smallcaps}
\begin{frame}{Small caps}
	This frame uses the \texttt{smallcaps} titleformat.

	\begin{alertblock}{Potential Problems}
		Be aware, that not every font supports small caps. If for example you typeset your presentation with pdfTeX and the Computer Modern Sans Serif font, every text in smallcaps will be typeset with the Computer Modern Serif font instead.
	\end{alertblock}
\end{frame}
}

{
\metroset{titleformat frame=allsmallcaps}
\begin{frame}{All small caps}
	This frame uses the \texttt{allsmallcaps} titleformat.

	\begin{alertblock}{Potential problems}
		As this titleformat also uses smallcaps you face the same problems as with the \texttt{smallcaps} titleformat. Additionally this format can cause some other problems. Please refer to the documentation if you consider using it.

		As a rule of thumb: Just use it for plaintext-only titles.
	\end{alertblock}
\end{frame}
}

{
\metroset{titleformat frame=allcaps}
\begin{frame}{All caps}
	This frame uses the \texttt{allcaps} titleformat.

	\begin{alertblock}{Potential Problems}
		This titleformat is not as problematic as the \texttt{allsmallcaps} format, but basically suffers from the same deficiencies. So please have a look at the documentation if you want to use it.
	\end{alertblock}
\end{frame}
}

\section{Elements}

\begin{frame}[fragile]{Typography}
      \begin{verbatim}The theme provides sensible defaults to
\emph{emphasize} text, \alert{accent} parts
or show \textbf{bold} results.\end{verbatim}

  \begin{center}becomes\end{center}

  The theme provides sensible defaults to \emph{emphasize} text,
  \alert{accent} parts or show \textbf{bold} results.
\end{frame}

\begin{frame}{Font feature test}
  \begin{itemize}
    \item Regular
    \item \textit{Italic}
    \item \textsc{SmallCaps}
    \item \textbf{Bold}
    \item \textbf{\textit{Bold Italic}}
    \item \textbf{\textsc{Bold SmallCaps}}
    \item \texttt{Monospace}
    \item \texttt{\textit{Monospace Italic}}
    \item \texttt{\textbf{Monospace Bold}}
    \item \texttt{\textbf{\textit{Monospace Bold Italic}}}
  \end{itemize}
\end{frame}

\begin{frame}{Lists}
  \begin{columns}[T,onlytextwidth]
    \column{0.33\textwidth}
      Items
      \begin{itemize}
        \item Milk \item Eggs \item Potatos
      \end{itemize}

    \column{0.33\textwidth}
      Enumerations
      \begin{enumerate}
        \item First, \item Second and \item Last.
      \end{enumerate}

    \column{0.33\textwidth}
      Descriptions
      \begin{description}
        \item[PowerPoint] Meeh. \item[Beamer] Yeeeha.
      \end{description}
  \end{columns}
\end{frame}
\begin{frame}{Animation}
  \begin{itemize}[<+- | alert@+>]
    \item \alert<4>{This is\only<4>{ really} important}
    \item Now this
    \item And now this
  \end{itemize}
\end{frame}
\begin{frame}{Figures}
  \begin{figure}
    \newcounter{density}
    \setcounter{density}{20}
    \begin{tikzpicture}
      \def\couleur{alerted text.fg}
      \path[coordinate] (0,0)  coordinate(A)
                  ++( 90:5cm) coordinate(B)
                  ++(0:5cm) coordinate(C)
                  ++(-90:5cm) coordinate(D);
      \draw[fill=\couleur!\thedensity] (A) -- (B) -- (C) --(D) -- cycle;
      \foreach \x in {1,...,40}{%
          \pgfmathsetcounter{density}{\thedensity+20}
          \setcounter{density}{\thedensity}
          \path[coordinate] coordinate(X) at (A){};
          \path[coordinate] (A) -- (B) coordinate[pos=.05](A)
                              -- (C) coordinate[pos=.10](B)
                              -- (D) coordinate[pos=.05](C)
                              -- (X) coordinate[pos=.10](D);
          \draw[fill=\couleur!\thedensity] (A)--(B)--(C)-- (D) -- cycle;
      }
    \end{tikzpicture}
    \caption{Rotated square from
    \href{http://www.texample.net/tikz/examples/rotated-polygons/}{texample.net}.
    }
  \end{figure}
\end{frame}
\begin{frame}{Tables}
  \begin{table}
    \caption{Largest cities in the world (source: Wikipedia)}
    \begin{tabular}{lr}
      \toprule
      City & Population\\
      \midrule
      Mexico City & 20,116,842\\
      Shanghai & 19,210,000\\
      Peking & 15,796,450\\
      Istanbul & 14,160,467\\
      \bottomrule
    \end{tabular}
  \end{table}
\end{frame}
\begin{frame}{Blocks}
  Three different block environments are pre-defined and may be styled with an
  optional background color.

  \begin{columns}[T,onlytextwidth]
    \column{0.5\textwidth}
      \begin{block}{Default}
        Block content.
      \end{block}

      \begin{alertblock}{Alert}
        Block content.
      \end{alertblock}

      \begin{exampleblock}{Example}
        Block content.
      \end{exampleblock}

    \column{0.5\textwidth}

      \metroset{block=fill}

      \begin{block}{Default}
        Block content.
      \end{block}

      \begin{alertblock}{Alert}
        Block content.
      \end{alertblock}

      \begin{exampleblock}{Example}
        Block content.
      \end{exampleblock}

  \end{columns}
\end{frame}
\begin{frame}{Math}
  \begin{equation*}
    e = \lim_{n\to \infty} \left(1 + \frac{1}{n}\right)^n
  \end{equation*}
\end{frame}
\begin{frame}{Line plots}
  \begin{figure}
    \begin{tikzpicture}
      \begin{axis}[
        mlineplot,
        width=0.9\textwidth,
        height=6cm,
      ]

        \addplot {sin(deg(x))};
        \addplot+[samples=100] {sin(deg(2*x))};

      \end{axis}
    \end{tikzpicture}
  \end{figure}
\end{frame}
\begin{frame}{Bar charts}
  \begin{figure}
    \begin{tikzpicture}
      \begin{axis}[
        mbarplot,
        xlabel={Foo},
        ylabel={Bar},
        width=0.9\textwidth,
        height=6cm,
      ]

      \addplot plot coordinates {(1, 20) (2, 25) (3, 22.4) (4, 12.4)};
      \addplot plot coordinates {(1, 18) (2, 24) (3, 23.5) (4, 13.2)};
      \addplot plot coordinates {(1, 10) (2, 19) (3, 25) (4, 15.2)};

      \legend{lorem, ipsum, dolor}

      \end{axis}
    \end{tikzpicture}
  \end{figure}
\end{frame}
\begin{frame}{Quotes}
  \begin{quote}
    Veni, Vidi, Vici
  \end{quote}
\end{frame}

{%
\setbeamertemplate{frame footer}{My custom footer}
\begin{frame}[fragile]{Frame footer}
    \themename defines a custom beamer template to add a text to the footer. It can be set via
    \begin{verbatim}\setbeamertemplate{frame footer}{My custom footer}\end{verbatim}
\end{frame}
}

\begin{frame}{References}
  Some references to showcase [allowframebreaks] \cite{knuth92,ConcreteMath,Simpson,Er01,greenwade93}
\end{frame}

\section{Conclusion}

\begin{frame}{Summary}

  Get the source of this theme and the demo presentation from

  \begin{center}\url{github.com/matze/mtheme}\end{center}

  The theme \emph{itself} is licensed under a
  \href{http://creativecommons.org/licenses/by-sa/4.0/}{Creative Commons
  Attribution-ShareAlike 4.0 International License}.

  \begin{center}\ccbysa\end{center}

\end{frame}

\begin{frame}[standout]
  Questions?
\end{frame}

\appendix

\begin{frame}[fragile]{Backup slides}
  Sometimes, it is useful to add slides at the end of your presentation to
  refer to during audience questions.

  The best way to do this is to include the \verb|appendixnumberbeamer|
  package in your preamble and call \verb|\appendix| before your backup slides.

  \themename will automatically turn off slide numbering and progress bars for
  slides in the appendix.
\end{frame}

\begin{frame}[allowframebreaks]{References}

  \bibliography{demo}
  \bibliographystyle{abbrv}

\end{frame}

\end{document}

